\section{Introduction  \label{section:introduction}}

% Description of what will be done 
In recent years Semi-Supervised learning has become a topic of great interest in the ML (Machine Learning) due to the dearth of labels in the huge amounts of data generated everyday. \\
In this tutorial we will show how to implement a simple Autoencoder neural network which will be used to dimensionally reduce a data set for more efficient semi-supervised learning. This will be demonstrated in \texttt{python} using the deep learning library \texttt{pytorch}, and the classification will be tested using the famous data set of handwritten digits,  \textbf{MNIST}. 


% Background for why this is important
\subsection{Background  \label{subsection:Background}}




% Tutorial Structure
\subsection{Tutorial Structure  \label{subsection:TutorialStructure}}
This adhere to the following structure, firstly we will go over some basic theory in relation to the implementation of Semi-Supervised learning.  This theory will cover motivations as to why this approach is effective, and cover the architectures that will be implemented in 

% Theory
% AutoEncoder Basics
% Perhaps some more detail

% Implementation in Pytorch

% Exercises along the way