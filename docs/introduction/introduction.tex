\section{Introduction  \label{section:introduction}}
% 100 Description of what we will be doing in this tutorial
Deep Generative models such as VAEs (Variational Autoencoders) and GANs (Generative Adversarial Networks), have garnered a large amount of interest in recent years due to their impressive performance on a range of image generation tasks.  Applications of these networks include the generation of photorealisic human images \cite{karras2018progressive}, or Google DeepMinds Draw network \cite{gregor2015draw}, which acts similar to a human artist  as it sequentially generates images. \\
In this tutorial we are going to use \href{https://www.python.org/}{python's} open sourced deep learning library \href{https://pytorch.org/}{pytorch} to apply a VAE  to the relatively simple task of generating digits from the  \href{http://yann.lecun.com/exdb/mnist/}{MNIST} database. This is a simple application of a VAE, however, it is  computationally cheap  and it demonstrates how images could be generated from more complicated data sets.\\




% Background for why this is important
%\subsection{Background  \label{subsection:Background}}
% 150  Words  - Slight history of autoencoders - why image generation is important 






\subsection{Tutorial Structure  \label{subsection:TutorialStructure}}
% 100 Words - Lay out the structure of the tutorial
The goal of this tutorial is to give the user enough, but not too much, theory and practical information such that they can begin experimenting with VAEs at its conclusion.\\
The tutorial is structured as follows,   \textbf{Section \ref{section:Theory}} will cover the  basic theory and architecture of VAEs. This section will show how VAEs differ from regular Autoencoders and focus on VAEs probabilistic interpretation.\\
The \texttt{pytorch} implementation will then be completed in \textbf{Section \ref{section:Implementation}}. Results from this implementation are presented in \textbf{Section \ref{subsection:Results}}.\\
During the tutorial there will be two theoretical and  two practical exercises for the user to  attempt. Solutions to the theoretical exercises can be found in \textbf{Section \ref{section:Exercise Solutions}}
